% ===========================================
% Individual Course Project (Undergraduate)
% Written by: Braidan Duffy
%
% Date: 07/27/2022
% Last Revision: 07/27/2022
% ============================================

\pagelayout{wide} % remove margins!

\setchapterstyle{koa}
\chapter{Individual Course Project: Undergraduate}
\setchapterpreamble[u]{\margintoc}
\labch{icp_ugrad}

\section*{Overview}
As part of this course, you are tasked with designing, ordering, and assembling your own Printed Circuit Board (PCB) to perform a measurement task.
This will be your first foray into the world of instrumentation design and analysis so it is important you consider your project choice carefully.
You only have a couple weeks to accomplish the goal, so consider your workload, capabilities, and experience when scoping out your ICP idea.

It is generally recommended that you consider taking one of the four previous projects and making that into your ICP.
For the most part, you already have the schematics and code finished, it is just up to the PCB design and assembly and what you want to do for that.
You can also make your PCB into an Arduino Uno-compatible shield, for a really slick looking product!

For those that are interested in more of a challenge, you may examine other projects you could do within your Arduino starter kit, or refer to ICP topics addendum.
There are several projects that will challenge you, but will be sponsored by the UTL outright and will probably be used within the lab for other projects or as a demonstration or teaching tool.

For completion of this project you must submit the following assignments on time, on the date specified in the syllabus and on Canvas, and as specified.

\section*{Assignment 1: Schematic}
Your first assignment will be to design a schematic for your ICP using industry standard practices, symbols, and nomenclature.
The lecture notes contain some examples and you are encouraged to look at companies like Adafruit for inspiration on layout, formatting, etc.
The instructor will also be available during office hours to offer advice and look over your schematic before you submit it.
It is highly encouraged you take advantage of the instructor's availability to get the best schematic possible.

Also keep in mind what components you have on hand.
Do not specify a component 

    \subsection*{Requirements}
    Your schematic must have the following:
    \begin{itemize}
        \item A clear frame sized for \emph{either} US Letter/ANSI A (8.5in x 11in) or US Tabloid/ANSI B (11in x 17in)
        \item Your name, project name, revision, and date created, as well other pertinent information within the title box of the frame.
        \item A well organized and logical layout for the parts used
        \item Clear and consistent nomenclature for the part values, net/bus names, and section names
        \item Effort towards presentability and readability (i.e. it should be easy to read!)
    \end{itemize}

    \subsection*{Submission}
    You will submit this assignment on Canvas before 23:59 on the date that it is due in a PDF format as well as the source schematic in an EAGLE-compatible format.
    Please consult the late policy found in the syllabus for additional details.

    \subsection*{Grading}
    You will be graded on the following criteria:

    \begin{table}[h!]
        \begin{tabular}{l | c}
            \toprule
            Criterion & Points \\

            \midrule
            Efficacy/Completeness & 40 \\
            Neatness and organization & 40 \\
            Feasibility & 10 \\
            Component usage & 10 \\
            \bottomrule
        \end{tabular}
    \end{table}

\section*{Assignment 2: Printed Circuit Board Design}
Your second assignment will be to take your schematic and turn it into a PCB for manufacturing.
This will involve component placement, layout, and wire routing, all of which can be intimidating so please make sure you leave yourself ample time to do the assignment right!
As always, consult the instructor during their office hours for assistance.
They are there to help you!

Here, it is important to try and keep things as neat as possible and make your routes direct and to the point to minimize potential issues.
This will be especially important when you are laying out components before routing them; don't place two components that need to be connected on opposite ends of the boards, unless necessary!

Before submission, run a Design Rules Check to ensure all of the connections have been made, there are not any overlaps or restricted-zone violations, or anything else that may jeopardize the manufacture process.
If you have any questions or concerns, please consult the instructor!

    \subsection*{Requirements}
    Your PCB design must have the following:
    \begin{itemize}
        \item A clear board that is of appropriate size (< 6in x 6in)
        \item Your name, revision, date of design (month/year), and project name (optional) in a silkscreen layer somewhere easily visible and unobstructed on the board.
        \item A well organized and logical layout for the parts used
        \item A well organized and thorough Bill of Materials for the parts used.
        \begin{itemize}
            \item Shall be in an Excel workbook format with the following headers: Item, Manufacturer, Part Number, Supplier, Supplier Part Number, Reference, Value, Package, Price, Qty Used, Component Price, Location, Datasheet, Notes
        \end{itemize}
        \item Gerber files for manufacture
        \item Effort towards aesthetics and user interactions
    \end{itemize}

    \subsection*{Submission}
    You will submit this assignment on Canvas before class on the date that it is due in a zipped archive containing the following:

    \begin{itemize}
        \item PDF containing the overall PCB design file, the top layers, and the bottom layers, all scaled to a readable amount
        \item Source file in an EAGLE-compatible format
        \item A complete Bill of Materials in Excel workbook format with the headers specified in the requirements
        \item Gerber files in a zipped archive as generated by your ECAD software
    \end{itemize}

    Please note that if this submission is not made on time, you risk your PCB not being ordered with enough time for delivery and assembly for the final report/presentation!

    \subsection*{Grading}
    You will be graded on the following criteria:

    \begin{table}[h!]
        \begin{tabular}{l | c}
            \toprule
            Criterion & Points \\

            \midrule
            Efficacy/Completeness & 30 \\
            Neatness and organization & 30 \\
            Bill of Materials & 30 \\
            Gerber files & 10 \\
            Extra Credit & 25 \\
            \bottomrule
        \end{tabular}
    \end{table}
    
    \subsection*{Extra Credit}
    There will be an extra credit opportunity with this assignment!
    If you want to, design a project logo or design that can be silk screened onto your PCB for some custom flair!
    Alternatively, design your PCb to be in a creative or unique shape.
    These points will be awarded subjectively by the instructor, but you must specify in your submission that you wish to have the extra credit points.

\section*{Assignment 3: Assembled Printed Circuit Board}
Your fourth assignment will be to assemble your PCB and test it!
You may assemble it by any means you desire.
For extra credit, you may program and use the pick and place machine located within the Underwater Technology Lab.
Be advised though, that machine is still in a prototype phase and may be difficult to program and use.
Please give yourself adequate time to learn it and use it.
Effort to use will still result in extra credit though, depending on how much time you dedicate to it.

Even though it may be your first time soldering, please ensure that all the solder joints are neat looking and the board is clean.
The PCB should be as presentable as possible!

If there are any issues, please consult the failure policy listed in the syllabus!

As always, please work with the instructor if you have any questions or concerns about assembly.
If you feel unsafe working with the soldering machines, talk to the instructor and something can be arranged.

    \subsection*{Requirements}
    Your PCB assembly must have the following:
    \begin{itemize}
        \item All components cleanly attached to the PCB
        \item Working PCB
    \end{itemize}

    \subsection*{Submission}
    You will bring your completed PCB to class on the day the assignment is due and perform a quick (~3 minute) show-and-tell for the instructor and your peers.
    Please have a suitable demonstration and talking points prepared!

    Please note that if this submission is not made on time, it will not be good for your grade.
    Even a non-functional ICP is better than nothing!

    \subsection*{Grading}
    You will be graded on the following criteria:

    \begin{table}[h!]
        \begin{tabular}{l | c}
            \toprule
            Criterion & Points \\
            \midrule
            Efficacy & 80 \\
            Neatness and organization & 10 \\
            Demonstration & 10 \\
            \bottomrule
        \end{tabular}
    \end{table}

\section*{Assignment 4: Assembled Printed Circuit Board}
Your fourth assignment will be to assemble your PCB and test it!
You may assemble it by any means you desire.
For extra credit, you may program and use the pick and place machine located within the Underwater Technology Lab.
Be advised though, that machine is still in a prototype phase and may be difficult to program and use.
Please give yourself adequate time to learn it and use it.
Effort to use will still result in extra credit though, depending on how much time you dedicate to it.

\emph{Note:} You should try to keep a comprehensive assembly and testing log for future reference!

Even though it may be your first time soldering, please ensure that all the solder joints are neat looking and the board is clean.
The PCB should be as presentable as possible!

If there are any issues, please consult the failure policy listed in the syllabus!

As always, please work with the instructor if you have any questions or concerns about assembly.
If you feel unsafe working with the soldering machines, talk to the instructor and something can be arranged.

    \subsection*{Requirements}
    Your PCB assembly must have the following:
    \begin{itemize}
        \item All components cleanly attached to the PCB
        \item Working PCB
    \end{itemize}

    \subsection*{Submission}
    You will bring your completed PCB to class on the day the assignment is due and perform a quick (~3 minute) show-and-tell for the instructor and your peers.
    Please have a suitable demonstration and talking points prepared!

    Please note that if this submission is not made on time, it will not be good for your grade.
    Even a non-functional ICP is better than nothing!

    \subsection*{Grading}
    You will be graded on the following criteria:

    \begin{table}[h!]
        \begin{tabular}{l | c}
            \toprule
            Criterion & Points \\
            \midrule
            Efficacy & 80 \\
            Neatness and organization & 10 \\
            Demonstration & 10 \\
            \bottomrule
        \end{tabular}
    \end{table}

\section*{Assignment 5: Final Presentation}
Your fifth assignment will be to give a comprehensive presentation on your ICP.
This presentation will be neatly formatted and cover the following topics, at a minimum:
    
    \begin{itemize}
        \item Purpose
        \item Design Considerations and Philosophy
        \item Assembly Notes (include build process)
        \item Testing Notes
        \item Conclusions
    \end{itemize}
    
    Remember to include citations and plenty of pictures, as needed!
        
        \subsection*{Submission}
        This will be submitted on Canvas in a PDF and/or PowerPoint-compatible form by the time of your ICP presentation.
        Please consult the late policy in the syllabus for more details.
    
        \subsection*{Grading}
        You will be graded on the following criteria:
    
        \begin{table}[h!]
            \begin{tabular}{l | c}
                \toprule
                Criterion & Points \\
                \midrule
                Purpose & 5 \\
                Design Process & 30 \\
                Testing Process & 30 \\
                Conclusion & 15 \\
                Presentation Style and Delivery & 10 \\
                Formatting & 10 \\
                \bottomrule
            \end{tabular}
        \end{table}

\section*{Assignment 6: Final Report}
Your sixth and final assignment will be to write a comprehensive report on your ICP.
This report will be neatly formatted and cover the following topics, at a minimum:

\begin{itemize}
    \item Executive Summary
    \item Introduction and Purpose
    \item Design Considerations and Philosophy
    \item Assembly Notes (include build process)
    \item Testing Notes
    \item Conclusions
\end{itemize}

Remember to include citations and plenty of pictures, as needed!
    
    \subsection*{Submission}
    This will be submitted on Canvas in a PDF form by 23:59 on the date specified in the syllabus.
    Please consult the late policy in the syllabus for more details.

    \subsection*{Grading}
    You will be graded on the following criteria:

    \begin{table}[h!]
        \begin{tabular}{l | c}
            \toprule
            Criterion & Points \\
            \midrule
            Executive Summary & 15 \\
            Purpose & 5 \\
            Design Process & 30 \\
            Testing Process & 30 \\
            Conclusion & 15 \\
            Formatting & 5 \\
            \bottomrule
        \end{tabular}
    \end{table}
        

\pagelayout{margin} % Restore margins
