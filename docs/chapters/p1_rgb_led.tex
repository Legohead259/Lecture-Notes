% ===========================================
% Project 1: RGB LED Cycler
% Written by: Braidan Duffy
%
% Date: 05/22/2022
% Last Revision: 05/25/2022
% ============================================

\chapter{Project 1: RGB LED Cycler}
\labch{p1_rgb_led}

\section*{Overview}
In this project, you will use the RGB LED within your Arduino kit to learn the basics of digital inputs, PWM output, switch-case statements, functions, and various loops.
The goal of this project will be to activate different modes on the RGB LED using a button to cycle through them and a switch-case statement to execute the appropriate functions.
Since this is a beginning project, there is some psuedocode below to give you an idea of how the code should be structured.
It will be up to you to determine what specific functions you want your LED to perform.

\section*{Requirements}
For this project, you must successfully hook up a button, an RGB LED, and any supporting passive components your Arduino and breadboard.
\marginnote{\textbf{Reminder:} The LED \emph{must} be connected to the Arduino pins with resistors in series in order to protect the diodes. Please consult your Arduino kit manual for specific resistor values}
You must also have at least three different functions that your LED performs - one of which must include loop. A great example would be the rainbow loop found in Example \ref{} or the breathe example found in Example \ref{} \todo{make these examples and put it in the book}

\section*{Submission}
You will be required to submit the following on Canvas:
\begin{enumerate}
    \item a video of the project working with narration of what is occurring
    \item a well-organized and documented schematic of the project setup
    \item the source code file
\end{enumerate}
Please package all of these items into a compressed (zipped) folder and upload them to Canvas.

\section*{Grading}
You will be graded along the following criteria:

\begin{table*}
    \begin{tabular}{ l | c }
        \toprule
        Criterion & Points \\

        \midrule
        Efficacy & 40 \\
        3 or more functions & 20 \\
        1 or more loops & 20 \\
        Schematic neatness & 10 \\
        Code neatness & 10 \\
        Mystery extra credit & 10 \\

        \bottomrule
    \end{tabular}
\end{table*}

\section*{Extra Credit}
If you are willing to dig in a little bit more, this project has a couple of opportunities to earn extra credit points at the discretion of the instructor.
Implementing something unique with the RGB LED or the button input will warrant the extra credit points.
\marginnote{\textbf{Hint:} Research the problem with button inputs}
If you want to try and get the extra credit points, please let the instructor know in the submission and detail why you believe you earn the points.

\section*{Psuedocode}

\begin{lstlisting}[linewidth=1.5\textwidth]
    Program: RGB LED Cycler

    Define BUTTON_PIN := [some pin]
    Define LED_RED_PIN := [some pin]
    Define LED_GREEN_PIN := [some pin]
    Define LED_BLUE_PIN := [some pin]
    Define MAX_LED_FUNCS := [number of LED functions]

    Initialize ledMode := 0

    Function: Setup
        Initialize Serial communication for debugging
        Initialize input pins
        Initialize output pins

    Function: Loop
        Check for button pressed
        If button is pressed then
            If ledMode is set to MAX_LED_FUNCS then
                Reset ledMode to 0
            Else
                Increment ledMode by 1
        Switch ledMode
            Case ledMode is set to 0
                Execute first LED function
            Case ledMode is set to 1
                Execute second LED function
            Case ledMode is set to 2
                Execute third LED function
            ...
    
    Function: First LED Function
        [YOUR CODE HERE]
    ...

\end{lstlisting}
