% ===========================================
% Data Processing
% Written by: Wensen Liu
%
% Date: 07/20/2022
% Last Revision: 07/20/2022
% ============================================

\setchapterstyle{kao}
\chapter{Software Basics and Architecture}
\setchapterpreamble[u]{\margintoc}
\labch{software_basics_and_arch}
% \addcontentsline{toc}{chapter}{Data Processing} % Add the preface to the table of contents as a chapter

While the world of physical hardware can be harnassed to develop systems that accomplish tasks, oftentimes these implementations become prohibitively complex as the scale of
implementation grows larger. For example, a modern computer processor contains billions of transistors and it would be impossible to set and reset each one of these transistors
by hand. Thankfully, as you probably already know, modern programming consists of several layers of abstraction above the transistor level, which allow us to develop meaningful
software solutions while leveraging the expansive amount of hardware that we have access to.

This chapter will focus on introducing you to the fundamentals of hardware-based software development. You may find that several if not all of these topics are also covered in
introductory programming courses. However, since we do not presume that everyone who has taken this class has had background programming experience, we will cover all the topics
necessary for this course from the ground up.

\section{Arduino Basics} \labsec{arduino_basics}
Lorem Ipsum Dolores

\section{Primitive Programming Datatypes} \labsec{primitive_programming_datatypes}
The baseline for every programming language is an understanding of datatypes.
Datatypes are how we can classify the characteristics of the blocks of information that our program will process.
These blocks of information can be collected data from the environment, information that the program stores internally, function inputs/outputs, etc.
Additional information on arduino datatypes (and in general, most arduino syntax) can be found on
\href{https://www.tutorialspoint.com/arduino/arduino_data_types.htm}{tutorialspoint} or the \href{https://www.arduino.cc/reference/en/}{Arduino Language Reference}
\marginnote{For the following sections, example code blocks will be given. As such, please note that the \enquote{$\backslash\backslash$} symbol is used to denote a \enquote{comment} and therefore
is not code that will execute and that the capitalization of keywords is deliberate (since Arduino C is case sensitive)}

    \subsection{void}
    The most basic of datatypes, the \textbf{void} keyword is used to represent \enquote{nothing}. For arduino programming, it is primarily used to denote that a function
    does not give us back any output. Do note, however, that the \textbf{void} datatype cannot be applied to variables.
    \marginnote{We will go more depth into functions in a later section, but for the purposes of this section you can think of functions as boxes that produce output from
    a given input. In the case of \textbf{void} we have no output.}
    
    \paragraph*{Proper usage of Void} The following code block demonstrates the proper usage of the \textbf{void} datatype.
    \begin{lstlisting}[linewidth=1.5\textwidth, language=C++]
        void function_name (\\ input variables) {
            \\ function code
            \\ note how there is no return statement (we are outputting nothing)
        }\end{lstlisting}

    \paragraph*{Improper usage of void} The following code block demonstrates the improper usage of the \textbf{void} datatype.
    \begin{lstlisting}[linewidth=1.5\textwidth, language=C++]
        void variable_name = \\ some value;\end{lstlisting}
    
    \subsection{bool}
    Boolean datatypes are used to store logic values, in this case the logic values of \textbf{True} or \textbf{False}. The \textbf{bool} keyword is used in both function and
    variable declarations to classify the function output or variable type as boolean logic.

    \paragraph*{Boolean usage} The following code block demonstrates how the \textbf{bool} keyword is used in a function declaration and a variable declaration
    \begin{lstlisting}[linewidth=1.5\textwidth, language=C++]
        bool function_name (\\ input variables) {
            bool variable_name = True;
            return variable_name;
        }\end{lstlisting}

    \subsection{Char}
    Character datatypes are special when programming in Arduino C. While they do represent what you would conventionally think as a character (for example, the letter \enquote{a}),
    they are defined based on their ascii value (\href{https://en.cppreference.com/w/cpp/language/ascii}{ASCII Reference}), which allow you to manipulate many more characters than
    what can be typed by a keyboard.

    The \textbf{Char} keyword is used in variable and function declarations to classify the result as a character, and there are two ways to syntaxically represent a character.
    \paragraph*{Method 1: Using Single Quotes} The first form of syntax makes more sense visually, and involves wrapping a typed character within two single quotes
    \begin{lstlisting}[linewidth=1.5\textwidth, language=C++]
        Char variable_name = 'a';\end{lstlisting}
    Do note, however, that this method restricts you to the characters that can be typed using a conventional keyboard. In addition, when declaring a character in this form, you
    cannot wrap multiple characters inside the quotes.
    \begin{lstlisting}[linewidth=1.5\textwidth, language=C++]
        Char variable_name = 'abc'; \\ this is improper declaration \end{lstlisting}
    \paragraph*{Method 2: Using ASCII Values} As mentioned previously, characters are encoded via ASCII values when used in a programming context. As such, we can define a
    character using its corresponding ASCII value.
    \begin{lstlisting}[linewidth=1.5\textwidth, language=C++]
        Char variable_name = 97; \\ ASCII value 97 corresponds to character 'a'\end{lstlisting}
    Since one character is equivalent to one byte of data, the maximum decimal value that can be used in a character declaration is 255.

    \begin{kaobox}[frametitle=Aside: Some quirks with characters]
        For all intents and purposes, characters are 8-bit binary numbers (0-255 in decimal). They become \enquote{characters} based on how they are interpreted when output to
        a display. Because of this dual interpretation, you can manipulate characters as if they were numbers. For example
        \begin{equation*}
            'a' + 1 = 'b'
        \end{equation*}
        However, if you were to use \enquote{1} (the character representation of 1) as the input you would get
        \begin{equation*}
            'a' + '1' = 97 + 49 = 156
        \end{equation*} 
        Which, for those that are curious, 156 represents the ASCII character \oe. In addition, there are special characters that can be typed using the single quotes 
        method that do not strictly follow the \enquote{only one character} rule. In particular, the $'\backslash n'$ (newline) and $'\backslash t'$ (tab) characters are often used.
    \end{kaobox}

    \section{unsigned char and byte}
    The unsigned character and byte datatypes are both datatypes that are used to represent a byte of data (8 bits, 0-255 in decimal). While they share the same range of values
    as the char datatype, they should not be used interchangeably with the char datatype. The \textbf{unsigned char} and \textbf{byte} keywords are used to define function output
    and variable values.
    \marginnote{Generally speaking, it is conventional to use the \textbf{byte} keyword over the \textbf{unsigned char} keyword for both brevity and code readability}
    \paragraph*{unsigned char and byte usage} the following block of code demonstrates how to use the \textbf{unsigned char} and \textbf{byte} keywords
    \begin{lstlisting}[linewidth=1.5\textwidth, language=C++]
        byte function_name (\\ input variables) {
            byte variable_name_1 = 0;
            unsigned_char variable_name_2 = 128;
            return variable_name_1;
        }\end{lstlisting}