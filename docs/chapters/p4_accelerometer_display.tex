% ===========================================
% Project 4: Accelerometer and LCD Display
% Written by: Braidan Duffy
%
% Date: 05/30/2022
% Last Revision: 06/13/2022
% ============================================

\chapter{Project 4: MPU6050 Accelerometer and LCD Display}
\labch{p4_accelerometer_display}

\section*{Overview} \labsec{p4_overview}
This project will introduce to you the concepts of reading sensor values and storing them in a packet, and displaying those values on an LCD display. 
You will read the accelerometer and gyroscope readings from the MPU6050 sensor (also referred to as the GY-521 module) and store them in an internal structured data packet for reporting to the LCD module.
Your LCD1602 module will have several different pages: one page will show the accelerometer values, another will show the gyroscope values, a third will show roll and pitch data, and others will show any additional information you desire.
\marginnote{See the \hyperref[sec:p4_extra_credit]{Extra Credit} section for more details.}
A button will be used to cycle between the different pages and a buzzer will be used to provide auditory feedback on every press.

\section*{Graduate Students} \labsec{p4_graduate_students}
You will have some additional work for this project. 
Since an accelerometer provides accelerations, it will be your task to extrapolate (integrate) velocity and position as well as the other tasks outline above.
You will be required to store this information in your storage packet and display the velocity and position values on separate pages on the LCD module.
Since the MPU6050 module lacks a magnetometer, it will also be your task to integrate the heading angle and save it to the telemetry packet.
You will also program the button to act as position and heading reset after you hold the button down for a certain amount of time.

\section*{Requirements} \labsec{p4_requirements}
For completion of this project, you must demonstrate the following:
\begin{outline}
    \1 Successful wiring of the MPU6050 IMU, LCD1602 module, piezoelectric buzzer, and button input (with appropriate debounce filtering)
    \1 Reading and calculating a variety of values from the MPU6050 sensor and storing them in a telemetry packet.
        \2 Acceleration X,
        \2 Acceleration Y,
        \2 Acceleration Z,
        \2 Gyroscope X,
        \2 Gyroscope Y,
        \2 Gyroscope Z,
        \2 Roll,
        \2 Pitch
    \1 Display the values on distinct pages on the PCD display with appropriate labels and units
        \2 Page 1: Acceleration values
        \2 Page 2: Gyroscope values
        \2 Page 3: Orientation values
    \1 Use a button tied to an interrupt service routine on the Arduino to cycle between the different LCD pages
    \2 Use a buzzer to provide auditory feedback on each button press 
\end{outline}

\section*{Submission}
You will be required to submit the following on Canvas:
\begin{outline}
    \1 a video of the project working with narration of what is occurring
    \1 a well-organized and documented schematic of the project setup
    \1 the source code file
\end{outline}
Please package all of these items into a compressed (zipped) folder and upload them to Canvas.

\section*{Grading} \labsec{p4_grading}
You will be graded on the following criteria:
\begin{table}[h!]
    \begin{tabular}{l | c}
        \toprule
        Criterion & Points \\

        \midrule
        Efficacy & 50 \\
        Well organized and neat schematic & 20 \\
        Well organized and neat source code & 30 \\
        Extra Credit Challenge 0 \footnotemark & 25 \\
        Extra Credit Challenge 1 & 25 \\
        Extra Credit Challenge 2 & 25 \\
        Extra Credit Challenge 3 & 25 \\

        \bottomrule
    \end{tabular}
\end{table}
\footnotetext{Graduate students: this will count towards your normal score}

\section*{Extra Credit} \labsec{p4_extra_credit}
There exists many opportunities for extra credit on this assignment. 
Ultimately, this is your chance to really explore and learn how to interact with sensors on a low level and gain an in-depth understanding of the relationship between microcontrollers and sensors.
To that end, it is highly encouraged to pursue the following challenges and earn as many points as possible.
A successful demonstration of \emph{all} of the challenges below will earn you \textbf{an exemption from one (1) non-ICP assignment}.

In your submission, please include a note of which challenges you have completed and videos of the challenge working. 
You will also need to include source code highlighting the specific challenge sections.
\emph{If it is not clear where your challenge-specific code is, it may not be counted.}

    \subsection*{Challenge 1: Filtering}
    Depending on what you set the update rate of the MPU6050 sensor, there may be a substantial amount of high frequency noise in your readings.
    There are a multitude of ways of dealing with this noise.
    For this challenge, implement a Butterworth filter on the Arduino to act as a low-pass digital filter for your data. 
    You need to tune your Butterworth filter to significantly attenuate noisy movements in your data, \textit{significantly} attenuating the good data signal.
    \marginnote{\emph{Hint:} You can view this \href{https://tttapa.github.io/Arduino-Filters/Doxygen/index.html}{link} for more information on getting started.}
    
    \textbf{FOR CREDIT:} Show two plots using the Arduino Serial plotter: one should show the unfiltered signal, the other should show the \textit{filtered} signal. 
    Only one axis should be shown.

    \subsection*{Challenge 2: Removing Gravity}
    This challenge will expand upon the filtering concpet introduced previously and show you how we can make gravity disappear! (At least in the sensor's frame of reference.)
    You will first need to calculate the angle between your board's reference frame and the Earth's reference frame.
    % \marginnote{\textit{Reminder:} Linear algebra shows this can be calculated with:
    %     \begin{math}
    %         A\dotB=|A||B|cos(\theta)
    %     \end{math}
    % Where A would be the Earth's acceleration vector [0 0 9.81] and B would be your recorded acceleration vector.}
    With the angle calculated, you will generate a \href{https://en.wikipedia.org/wiki/Rotation_matrix}{rotation matrix} and rotate your board acclerations to Earth-referenced accelerations.
    Then, simply remove Earth's acceleration vector from your measured accelerations and apply the inverse rotation matrix to get the measurements back into board-referenced space.
    This final value will be linear accelerations recorded by your accelerometer.
    Display this linear acceleration value on a page of your LCD module.

    \textbf{FOR CREDIT:} Show two plots using the Arduino Serial plotter of the board's Z-axis reading. 
    One plot should be the raw or filtered accelerations, the other should be the linear acceleration. 

    \subsection*{Challenge 4: Velocity, and Position}
    \marginnote{\textbf{Note:} This challenge is \emph{mandatory} for graduate students.}
    In this challenge, you are tasked with integrating the velocity, and position values from the accelerometer and gyroscope data.
    You must store these values in your telemetry packet and display them on separate pages on your LCD module.
    The button used to change display pages will also act as a position reset button when held for a certain amount of time.
    If you combine the previous challenges together, you should be able to get pretty accurate readings based on how you move your board.
    
    \textbf{FOR CREDIT:} In your submission video, find a ruler and place your board next to it with one of the acclerometer axes parallel to the straight edge, and the accelerometer chip centered on the ruler's 0 mark.
    Change the LCD to the position display page and reset the position.
    Push the board 6-10 inches along the ruler and stop, showing the position of the board and the reading it captured.
    \textit{Make sure you are constantly accelerating it for the best results!}
    
    \subsection*{Challenge 4: MPU6050 Motion Detect Interrupt}
    The interrupt pin on the MPU6050 can be configured to serve as a motion detection trigger.
    This challenge's task will be to put the Arduino into a sleep mode after the initialization and use the motion detection interrupt to wake the Arduino up and read sensor data for a defined period of time before going back to sleep.

    \textbf{FOR CREDIT:} In your submission video, demonstrate the Arduino not capturing any data (asleep), then you jolting the Arduino awake by moving the accelerometer, and the values changing on the LCD module.
    After a set time, these readings should stop updating as the Arduino goes back to sleep.
