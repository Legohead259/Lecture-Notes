% ===========================================
% Electrical Schematics
% Written by: Braidan Duffy
%
% Date: 07/18/2022
% Last Revision: 07/18/2022
% ============================================

\setchapterstyle{koa}
\chapter{Electrical Schematics and PCBs}
\setchapterpreamble[u]{\margintoc}
\labch{electrical_schemcatics}

One of the key pieces of instrumentation design is the electrical schematic and Printed Circuit Board (PCB).

\section{Schematic Basics}
\todo{Set chapter image to a schematic OR paste a big image of a goo schematic somewhere on this page}
The electrical schematic is a key piece of documentation that communicates to engineers what the circuit is comprised of and how it will work.
All modern electronic devices have schematics of varying complexity and depth that describe how electrons flow from one piece to another and document what behaviors can be expected during operations.
To understand these documents is to understand how a product fundamentally works, and to understand how to make these documents well is a skill that needs to practiced and refined by reviewing schematics and having yours reviewed.
For now, we will go over the basics and give you a fundamental understanding of what the schematic entails and what most of the symbols you might encounter mean.

    \subsection{Basic Notation}
    