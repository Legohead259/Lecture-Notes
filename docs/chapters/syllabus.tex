% ===========================================
% SYLLABUS
% Written by: Braidan Duffy
%
% Date: 05/22/2022
% Last Revision: 05/24/2022
% ============================================

\pagelayout{wide} % Remove margins
\chapter{Syllabus} 
\labch{syllabus}

\section*{Course Description} \labsec{course_desc}
This course is intended to give students exposure to practical electronic design and grant them additional programming experience. 
All of this will be accomplished through the lens of developing an instrumentation package that can be deployed as a project and capture/record data. 
Students will use basic Arduino learning kits to cultivate their interests. 
Graduate students taking the course will be required to follow more industry-related programming practices for the Arduino programming, will be held to a higher product quality standard, and will be required to develop a more advanced instrumentation package than the undergraduates. 
Course lectures will be supplemented by weekly (or as weekly as possible) demonstrations using real measurement equipment used by the Ocean Engineering department.

    \subsection*{Meeting Times}
        \subsubsection*{Lectures}
        Monday, Wednesday, Friday; 16:00-16:50 (0h50min)\\
        Fall 2022, August 22 - December 9\\
        Room TBD

        \subsubsection*{Office Hours}
        Monday, Wednesday, Friday; 13:00-15:00\\
        Or, by appointment (preferred)\\
        Link 155 or Frueauff 100

    \subsection*{Objectives \& Outcomes}
    The goal of this course is the provide a basic instruction on practical electrical engineering and computer programming through the lens of instrumentation design. 
    After introducing and refreshing basic concepts like electrical components, digital electronics, and C++ basics, students will be taught how industry professionals make real-world measurements using various instruments. At the end of the course, students will be able to:
    \begin{enumerate}
        \item Understand how various measurement devices capture, record, and transmit information to researchers and engineers
        \item Have a fundamental knowledge of programming Arduino-based microcontrollers
        \item Have a fundamental knowledge of Printed Circuit Board design, manufacture, and assembly
        \item Apply the fundamental concepts learned in lecture and through demonstrations on a practical Individual Course Project.
    \end{enumerate}

    \subsection*{Target Audience \& Prerequisites}
    On the undergraduate side, this course is intended for students who do not have a basic familiarity with basic electronics and programming with Arduino. 
    The ELEGOO Arduino Starter Kit recommended for this class has supplementary information that will assist students in understanding exactly what is occurring with various projects. 
    Students are strongly encouraged to delve deeper into the topics covered in class and pursue a challenging Individual Course Project. 
    Individuals with a strong drive to learn independently will benefit greatly during this course.

    On the graduate side, this course is designed for students who have a basic knowledge of electronics and programming with Arduino or C++. 
    Embedded design experience is also desired as it will be most beneficial during the Individual Course Project. 
    While the ELEGOO Arduino Starter Kit will give the basics, graduate students will be expected to expand upon the projects covered in the kit and use more sophisticated programming techniques. 
    The Individual Course Project will also need to be well considered and adequately demonstrate the student's knowledge and motivation to learn outside the classroom.

    \subsection*{Course Resources}
    The course material will be regularly updated on Canvas and \href{https://github.com/Legohead259/OCE4531-Material} {GitHub}. The GitHub repository will have all of the supplementary and study material required by students. The Canvas page may also contain these files, but GitHub will be the most up to date.
    
    Students will be \emph{required} to purchase their own Arduino learning kit for this course. They can be easily found on Amazon.
    \begin{enumerate}
        \item \href{https://www.amazon.com/ELEGOO-Project-Tutorial-Controller-Projects/dp/B01D8KOZF4}
        {UNO R3 Super Starter Kit - \$34}
        \item \href{https://www.amazon.com/EL-KIT-001-Project-Complete-Starter-Tutorial/dp/B01CZTLHGE} 
        {UNO R3 Complete Starter Kit - \$60}
        \item \href{https://www.amazon.com/EL-KIT-008-Project-Complete-Ultimate-TUTORIAL/dp/B01EWNUUUA}
        {Mega R3 Complete Ultimate Starter Kit - \$55}
    \end{enumerate}

    Graduate students will also be required to order a Raspberry Pi - or equivalent Single Board Computer (SBC), for this course. 
    It is strongly recommended to purchase a Raspberry Pi 400 Full Computer Kit (\href{https://www.adafruit.com/product/4796}{Adafruit}) with an accompanying T-Cobbler GPIO Breakout (\href{https://www.adafruit.com/product/2028}{Adafruit}).
    
    Due to the recent chip shortage, Raspberry Pis may be hard to find or prohibitively expensive. If students are unable to acquire their own Raspberry Pis or similar SBCs, they may ask for a loaner unit from the instructor. \textbf{Please note that if the loaner unit is \emph{not} returned by the end of the semester, the student will be given an ``Incomplete'' grade until the instructor has been given the device.}

\section*{Grading Policies}
This course covers several student performance metrics: (i) assignments, (ii) participation, (iii) midterm exam, and (iv) an Individual Course Project. The weighting for this metrics is below:
\begin{table*}[ht!]
    \begin{tabular}{c | c}
        \toprule
        Metric                      & Weight \\

        \midrule
        Assignments                 & 20\% \\
        Participation               & 10\% \\
        Midterm Exam                & 30\% \\
        Individual Course Project   & 40\% \\

        \bottomrule
    \end{tabular}
\end{table*}

Students will be assigned the following letter grade and GPA quality points based on their weighted sum assignment scores according to:

\begin{table*}[h!]
    \begin{tabular}{c | c | c}
        \toprule
        Score & Letter Grade & Quality Points \\
        
        \midrule
        90-100              & A     & 4 \\
        80-89               & B     & 3 \\
        70-79               & C     & 2 \\
        60-69\footnotemark  & D     & 1 \\
        <60                 & F     & 0 \\

        \bottomrule
    \end{tabular}
\end{table*}
\footnotetext{Undergraduate students only. \textbf{Graduate students will fail below a 70.}}

\section*{Course Schedule}
\begin{table*}[h!]
    \begin{tabular}{ c | c | c | c }
        \toprule
        Week & Monday & Wednesday & Friday \\

        \midrule
        1   & Syllabus, setup       & Binary and boolean logic          & Project discussion    \\
        2   & Digital logic         & Digital logic                     & YSI Castaway demo     \\    
        3   & \textbf{NO CLASS}     & Multiplexers, registers, states   & HOBO meter demo       \\
        4   & ADC Conversion        & Communication methods             & Survey gear demo      \\
        5   & Electrical comps      & Simple circuit debugging          & Launchsonde demo      \\
        6   & Intro to Fusion 360   & ECAD-Schematics                   & Lowell/Thetis demo    \\
        7   & ECAD-Schematics       & ECAD-PCB basics                   & Sidescan SONAR demo   \\
        8   & \textbf{NO CLASS}     & Inertial Measurement Units        & Midterm exam          \\
        9   & Sensor Fusion         & AHRS design                       & CODAR system demo     \\
        10  & Distance sensors      & Environmental sensors             & Soldering workshop    \\
        11  & Analog filtering      & Digital filtering                 & Remote sensing demo   \\
        12  & Gerber gen. and order & PCB manufacture techniques        & \textbf{NO CLASS}     \\
        13  & PCB assembly (hand)   & PCB assembly (PnP)                & Robotics demo         \\
        14  & Buffer day            & \textbf{NO CLASS}                 & \textbf{NO CLASS}     \\
        15  & Data transmission     & Power considerations              & Work day              \\
        16  & ICP Presentations     & ICP Presentations                 & ICP Presentations     \\

        \bottomrule
    \end{tabular}
\end{table*}

\section*{Assignment Schedule} \labsec{assignment_sch}

\begin{table*}[h!]
    \begin{tabular}{ c | c }
        \toprule
        Assignment & Due Date \\

        \midrule
        Get Arduino Kit, Raspberry Pi\footnotemark  & August 26     \\
        RGB LED                                     & August 29     \\
        Digital inputs and interrupts               & September 5   \\
        Active buzzer                               & September 12  \\
        Servo control, ICP proposal\footnotemark[2] & September 19  \\
        Ultrasonic sensor module                    & September 26  \\
        DHT temperature sensor                      & October 3     \\
        Midterm Exam                                & October 14    \\
        ICP schematic                               & October 24    \\
        ICP PCB design                              & November 11   \\
        Assembled ICP PCB                           & November 28   \\
        Final ICP presentation                      & December 5    \\
        Final ICP report                            & December 12   \\

        \bottomrule
    \end{tabular}
\end{table*}
\footnotetext[2]{\textbf{Graduate Students \emph{Only}}}

\section*{Other Important Dates}

\begin{table*}[h!]
    \begin{tabular}{ c | c }
        \toprule
        Event & Date \\

        \midrule
        Last day to drop without a "W"  & August 31 \\
        Last day to drop with a "W"     & October 31 \\

        \bottomrule
    \end{tabular}
\end{table*}

\section*{Course Policies} \labsec{course_policies}

    \subsection*{Online Course Management}

    This course will be published in the online learning tool, \href{instructure.fit.edu}{Canvas}, and will be made readily available to all students. 
    Canvas provides an online cross-platform solution for students and instructors to engage and will handle all of the assignment submissions and \emph{preliminary} grades for students.
    Assignments will be issued and submitted through the Canvas platform and students will be expected to submit the required documents by the due date and time listed on the assignment submission box.
    If, for whatever reason, assignments are unable to be turned in through Canvas, they must be emailed to the instructor and timestamped by the date and time established.
    
    Grades will also be posted on Canvas for students to track their progress in status in the course.
    However, there is no guarantee that the posted grade in Canvas will represent the final course grade submitted to the registrar, nor may it be up-to-date if grade corrections are necessary.
    In the event a student is not satisfied by their grade posted in Canvas, they are more than welcome to schedule a \emph{face-to-face} meeting with the instructor to discuss.
    Emailed requests to change grades may be considered but it will be more effective to meet with the instructor in-person to ensure the change is made properly.

    Canvas will also have a "Files" section where students can find relevant course resources and documents to aid in their studies.
    Students may request certain documents be uploaded to Canvas to share with their classmates and they may download all files freely if they wish to have their own local copies.
    The Canvas may be periodically updated to reflect changes or addendums to course content, but it may not necessarily reflect the most up-to-date information.
    For the most updated course material, please consult the class \href{https://github.com/OCE4531-Materials}{GitHub Repository}. \emph{You may be surprised what you find in there...}

    \subsection*{Weekly Projects}

    For the first couple of weeks of the course, students will be expected to put their Arduino starter kits to use with various projects.
    These are intended to slowly ramp up in complexity and give students a glimpse of what practical electronics and programming looks like.
    Projects will be due by 23:59 Eastern Time on the day specified in the \hyperref[sec:assignment_sch]{Assignment Schedule} section.
    
    \paragraph*{Late Policy} Submissions for assignments will be accepted up to the date of the students ICP presentation.
    However, for every day (24 hours) after the deadline, starting the minute after the deadline passes, 10 points will be deducted from the assignment down to an absolute maximum of 50\%.


    \subsection*{Individual Course Project}

    The Individual Course Project (ICP) is designed to encompass all of the elements taught throughout the course into a single package.
    Undergraduate students will be tasked with taking one of the Arduino projects available in their starter kits and converting it to an Arduino-compatible shield.
    Essentially, taking the circuit they already made on a breadboard, drawing the schematic in an ECAD software like Fusion 360, routing the PCB in the same software, and order and assembling the final PCB. 
    Students will be tasked with writing up their ICP in a comprehensive report and giving a quick final presentation at the end of the semester.

    \paragraph*{Financial Policy} Students will be required to purchase their PCBs and any additional components for their ICP not already available to them in their Arduino kits or by the Underwater Technology Lab. 
    Currently, each student is expected to pay around \$5 for the PCB order and most undergraduate students should be using components already present in their Arduino kits. 
    The final amount each student will have to pay for their PCB will be determined at the time of ordering by the instructor.
    
    Sponsorship by the UTL may be requested and may be granted on a case-by-case basis. Students with ICPs that prove useful to the lab's operation may have their fees waived depending on several external factors.
    This will be determined by the time of ordering.
    Students wishing to be sponsored should contact the instructor ASAP for details.

    \paragraph*{Failure Policy} \emph{FAILURE IS AN OPTION.} It is well-understood that students may have a non-functional ICP at the end of the semester. \textbf{That is okay!}
    The university environment is about learning and especially learning while failing.
    Therefore, students who do not have a functional PCB have two options for recourse:
    \begin{enumerate}
        \item they may explain, in copious detail, what the failure on the PCB was and how it would be addressed in the next revision, were it to be manufactured. 
        \item they will have until December 12 to submit a fixed PCB that is working, but may have "bodges" or other modifications
    \end{enumerate}
    If students elect for Option 1, they will receive penalty points on their ICP. These points will be subjectively detracted according to the instructor. As a general guideline, \emph{the simpler the mistake, the harsher the penalty will be}. 
    For instance, a missing connection in the schematic or trace on the PCB will have more points deducted than two components not talking due to EMI or other nuanced electrical engineering problem.
    \emph{Attention to detail is important!}

    \paragraph*{Late Policy} Late submissions on any part of the ICP will not be tolerated. For every 24 hours after the minute an ICP-related assignment is due, 10\% will be deducted from the \textbf{overall} ICP grade.
    Simply put, if four ICP-related assignments are turned in a minute late, then the student will receive a 0\% overall for their ICP weight category.
    
    Additionally, students who do not submit a PCB gerber package by the order deadline (November 11) may end up forfeiting their potential grades.
    This deadline is in place to ensure ample time for PCB assembly and testing and all orders will be placed at the same time.
    If a student does not submit the PCB gerber package by the deadline, the onus and financial responsibility will be on them to order the PCBs in a timely manner such that they can still assemble and test their ICP.

    \subsection*{For Students with Handicaps and/or Disabilities}
    Students with handicaps and/or disabilities will be given special considerations depending on their condition and Florida Tech policy. Please meet with the instructor privately to discuss any concerns or arrangements. 

    \subsection*{Academic Dishonesty Policy}
    Students who are caught cheating or plagiarizing will be given an audience with the instructor to discuss the situation. 
    Some cases are simple mistakes or coincidences with no ill-intent and can be rectified with a penalty to the assignment grade.
    Severe cases of rampant cheating or plagiarism \emph{with} ill-intent or severe negligence will be referred to the Dean of Students in accordance with academic policy present in the Florida Tech handbook.
    Students who are referred to the Dean of Students may forfeit their overall grades for the course and may face academic probation, suspension, or expulsion from Florida Tech - as the Dean determines.
    
    Academic Dishonesty incidents will not be considered until the assignment is officially submitted. 
    Therefore, students are strongly encouraged to meet with the instructor with questions about if a portion of their assignment could be considered academically dishonest.

    \subsection*{Title IX}
    Title IX of the Educational Amendments Act of 1972 is the federal law prohibiting discrimination based on sex under any education program and/or activity operated by an institution receiving and/or benefiting from federal financial assistance. Behaviors that can be considered “sexual discrimination” include sexual assault, sexual harassment, stalking, relationship abuse (dating violence and domestic violence), sexual misconduct, and gender discrimination. You are encouraged to report these behaviors. 

    \paragraph*{Reporting} Florida Tech can better support students in trouble if we know about what is happening.  Reporting also helps us to identify patterns that might arise - for example, if more than one complainant reports having been assaulted or harassed by the same individual.

    \subsection*{Regarding Unusual or Extraneous Circumstances}
    \emph{``If anything can go wrong, it will'' - Murphy's Law}

    It is well-understood that incidents will occur in this fickle thing we call life.
    Students and instructors alike are human and we cannot predict what will happen in the next five minutes let alone the next few days.
    In the event of something unexpected or unusual, please contact the instructor ASAP.
    Each case will be considered for its severity, impact on student well-being, and impact to the student's education.
    Some unforeseen circumstances may warrant an extension to an assignment deadline, a reschedule of a test, or other remedies depending on its severity.
    
    Conversely, the instructor may have incidents where class may need to be cancelled or an assignment delayed. The instructor reserves the right to maneuver the class as they see fit, but it would not be possible without the participation of the students.
    If the instructor is late to class, please allow for up to 15 minutes before departing.
    If the instructor needs to cancel a class, there will be an announcement made through the Canvas page, ideally, well ahead of the class start time.
    Additionally, the instructor may elect to hold class in a remote session through Zoom, should that be a necessary option.
    
    Students, please try to be flexible and understanding, and the instructors will do the same.

\pagelayout{margin} % Restore margins